\documentclass[a4paper,10pt]{article}
\usepackage{xcolor,geometry,amsmath,tikz}
\usepackage{everypage}

% Set page background to black and text color to white
\pagecolor{black}
\color{white}

% Reduce margins to use more space
\geometry{top=1.5cm, bottom=1.5cm, left=2cm, right=2cm}

% Load TikZ calc library
\usetikzlibrary{calc}

% Watermark configuration
\AddEverypageHook{%
  \begin{tikzpicture}[remember picture,overlay]
    \node[opacity=0.7,font=\small] at ($(current page.south east)+(-1.5cm,0.5cm)$) { Omar Abbas};
  \end{tikzpicture}%
}

\begin{document}

\section*{\centering Simple Discretization Methods: Binning}

\subsection*{1. Equal-Width (Distance) Partitioning}

\textbf{Formula:}
\[
W = \frac{B - A}{N}
\]
where:
\begin{itemize}
    \item \( W \) = Width of each interval  
    \item \( A \) = Minimum value in the dataset  
    \item \( B \) = Maximum value in the dataset  
    \item \( N \) = Number of bins  
\end{itemize}

\textbf{Example:}  
Given the dataset:  
\[
\{1, 3, 7, 10, 15, 18, 22, 25\}
\]
Let \( N = 3 \), \( A = 1 \), \( B = 25 \).

\textbf{Step 1: Compute Interval Width}
\[
W = \frac{25 - 1}{3} = \frac{24}{3} = 8
\]

\textbf{Step 2: Define Bins}
\begin{itemize}
    \item \textbf{Bin 1:} \([1, 9)\) → contains \(\{1, 3, 7\}\)
    \item \textbf{Bin 2:} \([9, 17)\) → contains \(\{10, 15\}\)
    \item \textbf{Bin 3:} \([17, 25]\) → contains \(\{18, 22, 25\}\)
\end{itemize}

This method is \textbf{simple and intuitive}, but \textbf{outliers} and \textbf{skewed data} can make it less effective.

\vspace{0.5cm}

\subsection*{2. Equal-Depth (Frequency) Partitioning}

\textbf{Concept:}  
Each bin should have \textbf{approximately the same number of elements}.

\textbf{Example:}  
Given the dataset:
\[
\{1, 3, 7, 10, 15, 18, 22, 25\}
\]
Let \( N = 3 \). Since there are \textbf{8 elements},  
each bin should have about \(\frac{8}{3} \approx 3\) elements.

\textbf{Step 1: Sort Data}  
\[
\{1, 3, 7, 10, 15, 18, 22, 25\}
\]

\textbf{Step 2: Assign Bins}
\begin{itemize}
    \item \textbf{Bin 1:} \(\{1, 3, 7\}\)
    \item \textbf{Bin 2:} \(\{10, 15, 18\}\)
    \item \textbf{Bin 3:} \(\{22, 25\}\)
\end{itemize}

This method ensures \textbf{equal distribution} of data points in each bin,  
making it \textbf{better for skewed data}.

\newpage % Start a new page

\section*{\centering Bin Processing Methods}

\subsection*{1. Bin Means}
\textbf{Concept:}  
Replace all values in a bin with the \textbf{mean} (average) of that bin.

\textbf{Example:}  
Using the same bins from equal-width partitioning:

\begin{itemize}
    \item \textbf{Bin 1:} \(\{1, 3, 7\}\) → Mean = \(\frac{1+3+7}{3} = 3.67\) → \(\{3.67, 3.67, 3.67\}\)
    \item \textbf{Bin 2:} \(\{10, 15\}\) → Mean = \(\frac{10+15}{2} = 12.5\) → \(\{12.5, 12.5\}\)
    \item \textbf{Bin 3:} \(\{18, 22, 25\}\) → Mean = \(\frac{18+22+25}{3} = 21.67\) → \(\{21.67, 21.67, 21.67\}\)
\end{itemize}

\textbf{Pros:} Reduces variance.  
\textbf{Cons:} May hide important variations in data.

\vspace{0.5cm}

\subsection*{2. Bin Medians}
\textbf{Concept:}  
Replace all values in a bin with the \textbf{median} of that bin.  
The \textbf{median} is the middle value when the numbers are sorted. If there is an even number of elements, the median is the average of the two middle values.

\textbf{Example:}
\begin{itemize}
    \item \textbf{Bin 1:} \(\{1, 3, 7\}\) → Sorted: \(\{1, 3, 7\}\), Median = \(3\) → \(\{3, 3, 3\}\)
    \item \textbf{Bin 2:} \(\{10, 15\}\) → Sorted: \(\{10, 15\}\), Median = \(\frac{10 + 15}{2} = 12.5\) → \(\{12.5, 12.5\}\)
    \item \textbf{Bin 3:} \(\{18, 22, 25\}\) → Sorted: \(\{18, 22, 25\}\), Median = \(22\) → \(\{22, 22, 22\}\)
\end{itemize}

\textbf{Pros:} Handles outliers better than means.  
\textbf{Cons:} May not capture overall distribution well.

\vspace{0.5cm}

\subsection*{3. Bin Boundaries}
\textbf{Concept:}  
Replace each value with the nearest boundary value in the bin.

\textbf{Example:}
\begin{itemize}
    \item \textbf{Bin 1:} \(\{1, 3, 7\}\) → Boundaries: \(1, 7\)  
          - \(1 \to 1\), \(3 \to 1\), \(7 \to 7\) → \(\{1, 1, 7\}\)
    \item \textbf{Bin 2:} \(\{10, 15\}\) → Boundaries: \(10, 15\)  
          - \(10 \to 10\), \(15 \to 15\) → \(\{10, 15\}\)
    \item \textbf{Bin 3:} \(\{18, 22, 25\}\) → Boundaries: \(18, 25\)  
          - \(18 \to 18\), \(22 \to 25\), \(25 \to 25\) → \(\{18, 25, 25\}\)
\end{itemize}

\textbf{Pros:} Good for preserving extreme values.  
\textbf{Cons:} Can distort middle-range values.

\newpage % Start a new page

\section*{\centering Data Transformation: Normalization}

\subsection*{1. Min-Max Normalization}

\textbf{Formula:}
\[
v' = \frac{v - \text{min}_A}{\text{max}_A - \text{min}_A} \times (\text{new\_max}_A - \text{new\_min}_A) + \text{new\_min}_A
\]
where:
\begin{itemize}
    \item \( v \) = Original value  
    \item \( \text{min}_A \) = Minimum value of the original dataset  
    \item \( \text{max}_A \) = Maximum value of the original dataset  
    \item \( \text{new\_min}_A \) = Minimum value of the new range  
    \item \( \text{new\_max}_A \) = Maximum value of the new range  
    \item \( v' \) = Normalized value
\end{itemize}

\textbf{Example:}  
Convert age = 30 to a range [0, 1], where \( \text{min} = 10 \) and \( \text{max} = 80 \):

\[
v' = \frac{30 - 10}{80 - 10} = \frac{20}{70} = \frac{2}{7} \approx 0.2857
\]

This transformation scales the value to fall between the new range, in this case, between 0 and 1.

\vspace{0.5cm}

\subsection*{2. Z-Score Normalization}

\textbf{Formula:}
\[
v' = \frac{v - \text{mean}_A}{\text{stand\_dev}_A}
\]
where:
\begin{itemize}
    \item \( v \) = Original value  
    \item \( \text{mean}_A \) = Mean of the original dataset  
    \[
    \text{mean}_A = \frac{1}{n}\sum_{i=1}^{n} x_i
    \]
    \item \( \text{stand\_dev}_A \) = Standard deviation of the original dataset  
    \[
    \text{Population: } \sigma = \sqrt{\frac{1}{n}\sum_{i=1}^{n} (x_i - \text{mean}_A)^2}
    \]
    \[
    \text{Sample: } s = \sqrt{\frac{1}{n-1}\sum_{i=1}^{n} (x_i - \text{mean}_A)^2}
    \]
    \item \( v' \) = Normalized value
\end{itemize}

The Z-score normalization transforms the data to have a \textbf{mean of 0} and a \textbf{standard deviation of 1}, making it useful for comparing data from different distributions.

\vspace{0.5cm}

\subsection*{3. Normalization by Decimal Scaling}

\textbf{Formula:}
\[
v' = \frac{v}{10^j}
\]
where \( j \) is the smallest integer such that:

\[
\text{Max}\left(|v'|\right) < 1
\]

This transformation scales the data by dividing each value by a power of 10, where the exponent \( j \) is chosen to ensure the maximum absolute value of \( v' \) is less than 1.

\vspace{0.5cm}

\newpage % Start a new page

\section*{\centering Five-Number Summary}

\subsection*{1. Even-Length Datasets}

\textbf{Formula:}
\[
\text{Five-number summary} = (\text{Minimum}, Q1, \text{Median (Q2)}, Q3, \text{Maximum})
\]
where:
\begin{itemize}
    \item \( Q1 \) = Median of the first half
    \item \( Q2 \) = Overall median (average of middle two values)
    \item \( Q3 \) = Median of the second half
\end{itemize}

\textbf{Example 1:}  
Given the dataset: \([1, 2, 3, 4, 5, 6, 7, 8, 9, 10]\)

\begin{itemize}
    \item \textbf{Minimum:} 1
    \item \textbf{Maximum:} 10
    \item \textbf{Median (Q2):} Average of 5th and 6th values = \(\frac{5+6}{2} = 5.5\)
    \item \textbf{Q1:} Median of \([1, 2, 3, 4, 5]\) = 3
    \item \textbf{Q3:} Median of \([6, 7, 8, 9, 10]\) = 8
\end{itemize}

\textbf{Five-number summary:} \((1, 3, 5.5, 8, 10)\)

\vspace{0.5cm}

\textbf{Example 2:}  
Given the dataset: \([1, 2, 3, 4, 5, 6, 7, 8, 9, 10, 11, 12]\)

\begin{itemize}
    \item \textbf{Minimum:} 1
    \item \textbf{Maximum:} 12
    \item \textbf{Median (Q2):} Average of 6th and 7th values = \(\frac{6+7}{2} = 6.5\)
    \item \textbf{Q1:} Median of \([1, 2, 3, 4, 5, 6]\) = \(\frac{3+4}{2} = 3.5\)
    \item \textbf{Q3:} Median of \([7, 8, 9, 10, 11, 12]\) = \(\frac{9+10}{2} = 9.5\)
\end{itemize}

\textbf{Five-number summary:} \((1, 3.5, 6.5, 9.5, 12)\)

\vspace{0.5cm}

\hrule
\vspace{0.5cm}

\subsection*{2. Odd-Length Datasets}

\textbf{Example 3:}  
Given the dataset: \([1, 2, 3, 4, 5, 6, 7, 8, 9]\)

\begin{itemize}
    \item \textbf{Minimum:} 1
    \item \textbf{Maximum:} 9
    \item \textbf{Median (Q2):} 5th value = 5
    \item \textbf{Q1:} Median of \([1, 2, 3, 4]\) = \(\frac{2+3}{2} = 2.5\)
    \item \textbf{Q3:} Median of \([6, 7, 8, 9]\) = \(\frac{7+8}{2} = 7.5\)
\end{itemize}

\textbf{Five-number summary:} \((1, 2.5, 5, 7.5, 9)\)

\vspace{0.5cm}

\textbf{Example 4:}  
Given the dataset: \([1, 2, 3, 4, 5, 6, 7, 8, 9, 10, 11]\)

\begin{itemize}
    \item \textbf{Minimum:} 1
    \item \textbf{Maximum:} 11
    \item \textbf{Median (Q2):} 6th value = 6
    \item \textbf{Q1:} Median of \([1, 2, 3, 4, 5]\) = 3
    \item \textbf{Q3:} Median of \([7, 8, 9, 10, 11]\) = 9
\end{itemize}

\textbf{Five-number summary:} \((1, 3, 6, 9, 11)\)

\newpage % Start a new page

\section*{\centering Statistical Formulas}

\subsection*{1. Median (Q2)}

\[
Q2 = \begin{cases}
x_{\frac{n+1}{2}} & \text{if } n \text{ is odd} \\
\frac{x_{\frac{n}{2}} + x_{\frac{n}{2}+1}}{2} & \text{if } n \text{ is even}
\end{cases}
\]

\vspace{0.5cm}

\subsection*{2. First Quartile (Q1)}

\[
p = \frac{n+1}{4}; \quad Q1 = \begin{cases}
x_p & \text{if } p \text{ is integer} \\
\frac{x_{\lfloor p \rfloor} + x_{\lfloor p \rfloor + 1}}{2} & \text{otherwise}
\end{cases}
\]

\textbf{Example:} For \(p = 2.3\), interpolate between \(x_2\) and \(x_3\).

\vspace{0.5cm}

\subsection*{3. Third Quartile (Q3)}

\[
p = \frac{3(n+1)}{4}; \quad Q3 = \begin{cases}
x_p & \text{if } p \text{ is integer} \\
\frac{x_{\lfloor p \rfloor} + x_{\lfloor p \rfloor + 1}}{2} & \text{otherwise}
\end{cases}
\]

\textbf{Example:} For \(p = 6.75\), interpolate between \(x_6\) and \(x_7\).

\vspace{0.5cm}

\subsection*{4. Mean}

\[
\text{Mean} = \frac{1}{n}\sum_{i=1}^{n} x_i
\]

\vspace{0.5cm}

\subsection*{5. Midrange}

\[
\text{Midrange} = \frac{\min(x) + \max(x)}{2}
\]

\vspace{0.5cm}

\subsection*{6. Range}

\[
\text{Range} = \max(x) - \min(x)
\]

\vspace{0.5cm}

\subsection*{7. Bin Width}

\[
\text{Bin Width} = \frac{\text{Range}}{k} \quad (k = \text{number of bins})
\]

\newpage % Start a new page

\section*{\centering Association Rules and Related Measures}

\subsection*{1. Support Count ($\sigma$)}
\textbf{Definition:} Frequency of occurrence of an itemset \(X\).

\textbf{Formula:}
\[
\sigma(X) = \text{Number of transactions containing } X
\]

\vspace{1cm}

\subsection*{2. Support ($s$)}
\textbf{Definition:} Fraction of transactions containing an itemset \(X\).

\textbf{Formula:}
\[
s(X) = \frac{\sigma(X)}{|T|}, \quad \text{where } |T| = \text{total number of transactions}
\]

\vspace{1cm}

\subsection*{3. Confidence ($c$)}
\textbf{Definition:} Conditional probability of a transaction containing \(Y\) given it contains \(X\).

\textbf{Formula (for rule $X \Rightarrow Y$):}
\[
c(X \Rightarrow Y) = \frac{\sigma(X \cup Y)}{\sigma(X)}
\]

\vspace{1cm}

\subsection*{4. Association Rule}
\textbf{Form:} \(X \Rightarrow Y\), where:
\begin{itemize}
    \item \(X \subset I\)
    \item \(Y \subset I\)
    \item \(X \cap Y = \emptyset\)
\end{itemize}

\textbf{Example:} \(\{x,y\} \Rightarrow z\)

\vspace{1cm}

\subsection*{Thresholds}
\begin{itemize}
    \item \textbf{Minimum Support (minsup):}\\
          Itemset \(X\) is ``frequent'' if \(s(X) \geq \text{minsup}\)
    \vspace{0.3cm}
    \item \textbf{Minimum Confidence (minconf):}\\
          Rule \(X \Rightarrow Y\) is valid if \(c(X \Rightarrow Y) \geq \text{minconf}\)
\end{itemize}

\vspace{1cm}

\subsection*{Notation Summary}
\begin{itemize}
    \item \(\sigma(X)\): Support count of itemset \(X\)
    \item \(|T|\): Total number of transactions
    \item \(X \cup Y\): Union of itemsets \(X\) and \(Y\)
    \item \(X \cap Y\): Intersection of itemsets \(X\) and \(Y\)
    \item \(X \subset I\): \(X\) is a subset of itemset \(I\)
\end{itemize}

\newpage % Start a new page

\section*{\centering Key Examples of Association Rules}

\subsection*{1. Support Calculation}
\[
s(\{\text{Milk, Diaper, Beer}\}) = \frac{\sigma(\{\text{Milk, Diaper, Beer}\})}{|T|} = \frac{2}{5} = 0.4
\]

\textbf{Explanation:}
\begin{itemize}
    \item Total transactions: \(|T| = 5\)
    \item Transactions containing \{\text{Milk, Diaper, Beer}\}: \(\sigma = 2\)
    \item Support is the fraction: \(\frac{2}{5} = 0.4\) or 40\%
\end{itemize}

\vspace{1cm}

\subsection*{2. Confidence Calculation}
\[
c(\{\text{Milk, Diaper}\} \Rightarrow \text{Beer}) = \frac{\sigma(\{\text{Milk, Diaper, Beer}\})}{\sigma(\{\text{Milk, Diaper}\})} = \frac{2}{3} \approx 0.67
\]

\textbf{Explanation:}
\begin{itemize}
    \item Transactions containing \{\text{Milk, Diaper, Beer}\}: \(\sigma = 2\)
    \item Transactions containing \{\text{Milk, Diaper}\}: \(\sigma = 3\)
    \item Confidence is the fraction: \(\frac{2}{3} \approx 0.67\) or 67\%
\end{itemize}

\end{document}
